%! Date = 18/07/2022

\documentclass[a4paper,11pt]{article}

\usepackage{amsmath}
\usepackage[ngerman]{babel}
\usepackage{relsize}
\usepackage{caption}
\usepackage{tabularx}
\usepackage{minted}

\newcommand{\heading}[1]{\multicolumn{1}{c}{#1}}

\author{Adrian Schurz}
\title{Projektantrag\\[0.2em]\smaller{}Terminal-Config-Manager}

\begin{document}

\maketitle
\pagenumbering{gobble}
\newpage
\pagenumbering{arabic}
\tableofcontents

\section{Einführung}
\paragraph{}
Dieses Dokument dient als Projektantrag im Rahmen der externen Zulassung
zur Abschlussprüfung einer Ausbildung zum Fachinformatiker für
Anwendungsentwicklung bei der IHK Dresden. Es beschreibt das Ergebnis der
Anforderungsanalyse des gewählten Softwareprojekts. Dazu gehören die
Motivation des Projekts, die Anforderungen sowie Nicht-Anforderungen an
die Software als auch die prüfbaren Kriterien anhand derer die Abnahme des
fertigen Produkts zu erfolgen hat.

\section{Motivation}
\paragraph{}
Bei der Arbeit an Software-Anwendungen, welche eng mit einer Vielzahl anderer
solcher Anwendungen über mehrere Umgebungen hinweg interagieren, ergaben sich
zwei häufige Anwendungsfälle (\textbf{A} und \textbf{B}) beim Umgang mit
Konfigurationsdateien.
\paragraph{A}
Das Nachschlagen gezielter Werte in den Konfigurationsdateien
einzelner Anwendungen oder Hilfsprogramme auf einem lokalen, zur Entwicklung
genutzten Computer.
\paragraph{B}
Das Umschalten einer kleinen Menge häufig zu ändernder Konfigurationswerte
zwischen verschiedenen Werten. Die Anzahl verschiedener Zielwerte ist dabei
meist sehr begrenzt.
\paragraph{}
Diese beiden häufigen Arbeitsvorgänge sollen so weit wie möglich beschleunigt
werden.

\section{Beispielszenario}
In der Continuous Integration (CI) - Pipeline einer zentralen Anwendung schlagen
unvermittelt Akzeptanztests fehl. Um unverzüglich die Ursache feststellen zu
können muss das lokale Setup, das in diesem Moment an die Entwicklung einer
anderen Anwendung und für die Ausführung einer anderen Menge an Tests
konfiguriert ist, umkonfiguriert werden.

\section{IST-Zustand}
\paragraph{}
Um die Zielanwendung zu untersuchen sind viele Teilaufgaben zu erledigen,
beispielsweise das Aktivieren detaillierter Lognachrichten, das Ändern der
auszuführenden Unit-/Integration-/Systemtestsuites, das Ändern von Cache-Verhalten,
das anpassen von Zielhosts bzw. -umgebungen anderer, an der Testausführung
beteiligter Anwendungen, usw.. Jeder dieser Schritte ist wiederum mit mehreren
Einzelschritten verbunden: Meist muss zu diesem Zweck eine passende IDE gestartet,
die relevante Konfigurationsdatei ermittelt und zum richtigen Unterordner navigiert
werden. Danach muss die Datei geöffnet, ihr Inhalt untersucht und der Zielwert
geprüft und gegebenenfalls angepasst werden. In Summe sind sehr viele,
repetitive Einzelaufgaben händisch zu erledigen. Hier bietet sich daher Potential
zur Verbesserung.

\section{SOLL-Zustand}
\paragraph{}
Es existiert ein Kommandozeilenprogramm das übersichtlich eine Liste von
Anwendungsfällen anzeigt. Ein Anwendungsfall ist in diesem Fall eine kurze
Beschreibung des von einem Konfigurationswert kontrollierten Verhaltens zusammen
mit dessen aktuellem Wert. Das Programm erlaubt es zwischen diesen
Anwendungsfällen per Tastendruck auszuwählen und den zugehörigen Wert aus einer
weiteren Liste möglicher Werte auszuwählen. Wird ein neuer Wert ausgewählt so
wird die dazugehörige Konfigurationsdatei entsprechend umgeschrieben. Sowohl die
Zielkonfigurationsdateien als auch die jeweiligen möglichen Werte sind
konfigurierbar. Zusätzlich existieren ausführliche Unit- und Akzeptanztests.

\section{Projektumfeld}
\paragraph{}
Die CHECK24 Vergleichsportal GmbH und CHECK24 Vergleichsportal Reise GmbH sind
Betreiber von check24.de, einer Website auf der verschiedene Produkte zum Vergleich
angeboten werden. Der Auftraggeber CHECK24 Vergleichsportal Reise GmbH betreibt
auf dieser Online-Plattform eine Vergleichsmöglichkeit von Pauschalreisen.
\paragraph{}
Die Menge und Komplexität interner Anwendungen, die am Produktionsbetrieb
und der Qualitätssicherung beteiligt sind, wächst stetig. Damit einher gehen
komplexere Interaktionen und Konfigurationsmöglichkeiten, die häufige
Fehlerquellen im Betrieb und während der Entwicklung darstellen. Je einfacher
diese Konfiguration möglich ist, desto schneller kann während der Fehlersuche
und der Entwicklung gearbeitet werden.

\section{Projektphasen}

% \caption{Nutzungsszenarien und Vergleich der Arbeitsabläufe}

\begin{tabularx}{0.8\textwidth}{ c | c }
	\textbf{Phase}                    & \textbf{Dauer (h)} \\ \hline
	Konzeption                        & 10                 \\ \hline
	Wahl des Techstacks               & 3                  \\ \hline
	Einrichtung, Entwicklungsumgebung & 2                  \\ \hline
	Implementierung                   & 35                 \\ \hline
	Qualitätssicherung                & 15                 \\ \hline
	Dokumentation                     & 5                  \\ \hline
	Gesamt                            & 70
\end{tabularx}


\section{Nicht-Ziele}
\paragraph{A}
Das automatische Ausführen von Folgeaktionen nach Änderung eines Wertes. \textbf{Begründung:}
Die möglichen Arten von Folgeaktionen, bspw. das (Neu-)Starten von Anwendungen,
das Löschen von Cachedateien, usw., sind so vielfältig, dass zu diesem Zweck vom
Programm beliebige Shellskripte o.ä. ausgeführt werden müssten. Das erhöht zum Einen
unnötigerweise die Komplexität des Programms und zum Anderen wird die Betrachtung von
Sicherheitsaspekten wesentlich schwieriger. Um Folgeaktionen auszuführen exisiteren
eine Vielzahl spezialisierter Tools die jeweils einem begrenzten Aufgabenbereich
gerecht werden und sich gut miteinander kombinieren lassen.

\begin{figure}[!ht]
	\caption[cdba]{Beispiel zum Ausführen einer beliebigen Folgeaktion nach
		Änderung an einer Konfigurationsdatei in bash\footnotemark mittels entr.\footnotemark}
	\begin{minted}{bash}
    $ ls <Konfigurationsdateipfad> | entr Folgeaktion.sh
\end{minted}
\end{figure}
\footnotetext[1]{https://www.gnu.org/software/bash/}
\footnotetext[2]{https://eradman.com/entrproject/}

\paragraph{B}
Das gleichzeitige Ändern mehrerer Konfigurationswerte in festgelegten Kombinationen
oder "Profilen". \textbf{Begründung:} Wenn mehrere voneinander abhängige Konfigurationswerte
existieren, die beim Wechsel hin zu einem anderen Nutzungsszenario immer gleichzeitig
geändert werden müssen, dann ist dies im SOLL-Zustand ebenfalls mit wenig Arbeit
verbunden. Zusätzlich exisitert die Option die Konfiguration der Anwendung zu
verbessern und verschiedene Szenarien mit einzelnen Konfigurationswerten abzubilden.
\paragraph{C}
Ein grafisches Benutzerinterface - dazu gehören Webinterfaces. \\
\textbf{Begründung:} Für die Darstellung und Benutzerinteraktion ist ein
textbasiertes Terminal völlig ausreichend. Die Komplexität des Programmcodes und
die Abhängigkeiten des Projekts für eine (Web-)Oberfläche um mehrere
Größenordnungen zu vergrößern ist nicht ohne Weiteres gerechtfertigt.

\section{Dokumentation zur Projektarbeit}
Das Projekt wird ausführlich dokumentiert werden. Dazu gehören
\begin{itemize}
	\item eine ausführliche, aus Quellcodekommentaren generierte, Dokumentationen
	      der einzelnen Programmmodule im HTML-Format
	\item Hilfetexte, die vom Programm selbst bei Bedarf ausgegeben werden.
	\item README-Dateien, welche das Aufsetzen des Projekts, die Kompilierung
	      und die Ausführung der Unit- und Akzeptanztests beschreiben
	\item Eine Projektdokumentation im PDF-Format, welche eine IST-Analyse,
	      die Anforderungen, die Projektziele, die Zeitplanung, den Projektverlauf
	      und das Pflichtenheft beinhaltet
\end{itemize}

\section{Präsentationsmittel}

\end{document}

% TODO Nichtziel B besser begründen
% TODO Überlegung: Wert bei jedem switch ändern vs erst nach Enter
% TODO evaluieren ob Ändern von ETCD-Werten sinnvoll ist. -> ist es nicht wegen Latenz