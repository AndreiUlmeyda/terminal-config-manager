\section{Ergebnisdiskussion} \label{Ergebnisdiskussion}

\subsection{Funktionalität}
Die Hauptfunktionalität des Programms wurde erfolgreich umgesetzt und es wird von
mir selbst bereits genutzt. Das Programm arbeitet seither wie erwartet,
ist konfigurierbar und informiert mit lesbaren Nachrichten im Falle eines Fehlers.
Ein weiteres Feature wäre jedoch wünschenswert, die Fähigkeit des Programms mit
schreibgeschützten Zieldateien umgehen zu können und in diesem Fall eine
sudo-Passwortabfrage auszulösen. Letzteres ist nicht Teil der
ursprünglichen Anforderungen und wird daher erst zukünftig umgesetzt.

\subsection{Domain Driven Design} \label{domain-driven-design-result}
Eines der Nebenziele war es, die Abhängigkeiten der einzelnen Softwaremodule
untereinander einem speziellen Schema folgen zu lassen (siehe Paragraf
Softwaredesign in Kapitel \ref{par:softwaredesign}). Die ist mit einer Ausnahme
geglückt. Im Zielschema im Anhang, Abb. \ref{domain-driven-design-layers}, sind Module
der Applikationsebene ausschließlich abhängig von Modulen des User-Interface.
Das verwendete \gls{GUI}-Framework, Brick \cite{brick}, führt durch sein Interface allerdings
zu einem Muster, bei dem diese Abhängigkeit umgekehrt ist. Letzteres wird deutlich
beim Vergleich des Schemas mit den tatsächlichen Modulabhängigkeiten (siehe Abb.
\ref{module-dependency-graph} im Anhang). Dieser Umstand stellt für den Moment
eine vernachlässigbare Designschwäche dar die ohne weitere
Konsequenz ist. Aus diesem Grund erhielt die Aufgabe dies zu beheben eine niedrige
Priorität und steht vorerst noch aus. Alle sonstigen Module halten das angestrebte
Schema ein.

\subsection{Moduldokumentation}
Das Ziel die Moduldokumentation stetig während des Buildprozesses aktuell zu
halten ist teilweise geglückt. Im Projektverzeichnis liegt die ausführliche Moduldokumentation,
einschließlich jener der eingebundenen Softwarebibliotheken im HTML-Format vor.
Auszüge davon sind in den Abbildungen \ref{module-doc-index} und \ref{module-doc-state} dargestellt.
Allerdings gibt es seit wenigen Tagen bei neueren Versionen des Buildsystems Stack \cite{stack}
und des Generierungstools Haddock \cite{haddock} eine Inkompatibilität. Bis diese
in diesen Projekten behoben und veröffentlicht ist, ist die in diesem Projekt hinterlegte
Moduldokumentation nicht aktuell. Es ist zu erwarten, dass dieses Problem in naher
Zukunft seitens der Entwickler der beiden Tools behoben wird.

\subsection{Testabdeckung} \label{Testabdeckung}
Die Abdeckung der Funktionalität des Programms auf Unittest-Ebene ist, besonders
dank der Verwendung von \gls{Property-Based-Testing} \cite{property-based-testing}
zufriedenstellend. Einen groben Überblick über die Unittests der einzelnen Module
gibt eine Beispielausgabe der Ausführung der Testsuite in Abb. \ref{unit-test} im
Anhang.

\paragraph{}
Eines der angestrebten Ziele war jedoch neben ausführlichen Unittests ebenso ausführliche
Akzeptanztests bereitzustellen. Dies ist trotz erheblichem Aufwand gescheitert.
Das Programm ist eine Konsolenanwendung, welche auf Tasteneingaben reagiert. Das
bedeutet, dass automatisierte Tests in einer kontrollierten Umgebung Terminal-Emulatoren
starten und ihnen Tasteneingaben simulieren müssen, um das Programm zu testen. Je nach
der verwendeten grafischen Benutzeroberfläche des Betriebssystems unterscheiden sich
die Ansätze dies zu erreichen jedoch stark. Es existieren mehr oder weniger gut
gepflegte Open-Source-Softwaretools für die Teilaufgabe der Eingabeemulation. Verschiedene
Kombinationen dieser Tools (xdotool \cite{xdotool} vs. ydotool \cite{ydotool}) wurden
mit verschiedenen Displayservern (xorg \cite{xorg} vs. wayland \cite{wayland}),
Betriebssystemen (Ubuntu \cite{ubuntu} vs. Arch Linux \cite{arch}), und
Virtualisierungslösungen (virtualbox \cite{virtualbox} vs. docker \cite{docker})
evaluiert. Keine der Varianten führte zum Erfolg.

\subsection{Bekannte Fehler}
Es sind zwei Bugs in Software-Abhängigkeiten des verwendeten \gls{GUI}-Frameworks
bekannt, welche mit geringer Häufigkeit das \gls{Rendering} bzw. den Start des Programms
beeinträchtigen. Diese Bugs sind dokumentiert (siehe \cite{bug-vty-startup-crash} und
\cite{bug-vty-terminal-capabilities}). Einer dieser Fehler führt dazu, dass der
aktuelle Wert eines Eintrags weiß statt blau gerendert wird und ist schwer
reproduzierbar (ca. 1 von 50 bis 100 Programmstarts). Der andere Fehler führt zu
einem Crash des Programms beim Start (ca. 1 von 30 bis 50 Programmstarts).

Beide Softwarefehler befinden sich in eingebundenen Open-Source-Softwarebibliotheken.
Bugfixes dieser Projekte werden, sobald sie verfügbar werden, für dieses Projekt
übernommen.