\makeglossaries

\newglossaryentry{Textpassage}
{
    name=Textpassage,
    description={Ein Stück Text aus einer Textdatei. Dabei
            wird sich meist auf eine in der Konfigurationsdatei spezifizierte
            Zieldatei des Programms bezogen.}
}

\newglossaryentry{Zielmuster}
{
    name=Zielmuster,
    description={Ein Element innerhalb der Konfigurationsdatei. Es bezeichnet ein
            Stück Text das einen speziellen \gls{Platzhalter}, den Text \mintinline{bash}{{{value}}},
            enhält. Es wird vom Programm dazu genutzt den genauen Ort der Ziel-\gls{Textpassage}
            innerhalb der Zieldatei zu identifizieren. Das Zielmuster sollte}
}

\newglossaryentry{Platzhalter}
{
    name=Platzhalter,
    description={Ein vordefinierter Text: das englische Wort \mintinline{bash}{value}
            umgeben von doppelten geschweiften Klammern:
            \mintinline{bash}{{{value}}} Das Format des Platzhalters ist an
            die Template-Engine Template-Engine Jinja2 \cite{jinja2} angelehnt.
            Er ist Teil des \gls{Zielmuster}s das zusätzlich Text vor und nach
            der Ziel-\gls{Textpassage} enthält. Er markiert den Ort der Ziel-\gls{Textpassage}
            relativ zum \gls{Zielmuster} innerhalb der Zieldatei.}
}

\newglossaryentry{Property-Based-Testing}
{
    name=Property-Based-Testing,
    description={Property-Based-Testing bezeichnet eine spezielle Strategie Softwaretests
            zu formulieren. Statt einer Reihe explizit ausformulierter Tests wird eine Regel
            formuliert, welche das zu testende Programm für automatisch generierte Eingabewerte
            einhalten muss. Die in großer Zahl automatisch generierten Testfälle
            decken mit höherer Wahrscheinlichkeit Grenzfälle durch z.B. besonders
            große oder abwegige Eingabewerte ab.}
}

\newglossaryentry{GUI}
{
    name=GUI,
    description={Abkürzung für den Begriff graphical user interface. Zu deutsch:
            grafische Benutzeroberfläche.}
}

\newglossaryentry{Rendering}
{
    name=Rendering,
    description={Im Kontext dieses Programms beschreibt der Begriff Rendering den
            Prozess der Wandlung eines Zustands des Programms in eine auf dem Bildschirm
            darstellbare Form, in diesem Fall Text.}
}

\newglossaryentry{IDE}
{
    name=IDE,
    description={Abkürzung für den englischen Begriff integrated development
            environment - integrierte Entwicklungsumgebung}
}

\newglossaryentry{PDF}
{
    name=PDF,
    description={Abkürzung für den englischen Begriff portable file format. Es
            bezeichnet ein weit verbreitetes Dokumentformat.}
}

\newglossaryentry{Terminalemulator}
{
    name=Terminalemulator,
    description={"Ein Terminalemulator ist ein Computerprogramm, das die Funktion
            eines Computer-Terminals nachbildet. Sie wird genutzt, um textbasierte Programme
            innerhalb einer grafischen Benutzeroberfläche verwenden zu können." -
            abgewandelter Auszug aus \cite{terminal-emulator}}
}

\newglossaryentry{Betatesting}
{
    name=Betatesting,
    description={"Der Begriff Betatest bezeichnet den Softwaretest eines Software-Produktes
            im Entwicklungsstadium einer Beta-Version, der unter möglichst realen Anwendungssituationen
            von späteren Benutzern („Nachfrager“) durchgeführt wird." -
            abgewandelter Auszug aus \cite{beta-testing}}
}

\glsaddall