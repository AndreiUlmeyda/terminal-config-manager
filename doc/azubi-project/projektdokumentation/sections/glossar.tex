\makeglossaries

\newglossaryentry{Textpassage}
{
    name=Textpassage,
    description={Ein Stück Text aus einer Textdatei. Dabei
            wird sich meist auf eine in der Konfigurationsdatei spezifizierte
            Zieldatei des Programms bezogen.}
}

\newglossaryentry{Zielmuster}
{
    name=Zielmuster,
    description={Ein Element innerhalb der Konfigurationsdatei. Es bezeichnet ein
            Stück Text das einen speziellen \gls{Platzhalter}, den Text \mintinline{bash}{{{value}}},
            enhält. Es wird vom Programm dazu genutzt den genauen Ort der Ziel-\gls{Textpassage}
            innerhalb der Zieldatei zu identifizieren. Das Zielmuster sollte}
}

\newglossaryentry{Platzhalter}
{
    name=Platzhalter,
    description={Ein vordefinierter Text: das englische Wort \mintinline{bash}{value}
            umgeben von doppelten geschweiften Klammern:
            \mintinline{bash}{{{value}}} Das Format des Platzhalters ist an die
            die Template-Engine \cite{template-engine} Jinja2 \cite{jinja2} angelehnt.
            Er ist Teil des \gls{Zielmuster}s das zusätzlich Text vor und nach
            der Ziel-\gls{Textpassage} enthält. Er markiert den Ort der Ziel-\gls{Textpassage}
            relativ zum \gls{Zielmuster} innerhalb der Zieldatei.}
}

\glsaddall