\section{Textteil - vorläufiger Titel} \label{Textteil}
\subsection{Problemstellung}
\paragraph{}
Der Bedarf für die im Rahmen dieser Projektarbeit erstellte Softwarelösung ergab
sich bei der Arbeit an Software-Anwendungen, welche eng mit einer Vielzahl
anderer solcher Anwendungen über mehrere Umgebungen hinweg interagieren.

\paragraph{}
Der Großteil dieser Anwendungen besitzt weitläufige Konfigurationsmöglichkeiten
welche ihren Betrieb in verschiedensten Szenarien steuern. Beispiele für
Konfigurationsmöglichkeiten und deren Ausprägungen sind:

\begin{itemize}
    \item Das Loggingverhalten der Anwendung
          \begin{itemize}
              \item Logging gegen die Logverarbeitungssoftware der Produktionsumgebung
              \item Logging gegen eine lokale Instanz der Logverarbeitungssoftware
              \item Logging auf das Dateisystem
              \item Logging mit verschiedenen Logleveln
          \end{itemize}
    \item Die Zieldatenbank der Anwendung \begin{itemize}
              \item Datenbank der Produktionsumgebung
              \item Datenbank der Testumgebung
              \item lokale Datenbank
          \end{itemize}
    \item Die Ausführung von Softwaretests \begin{itemize}
              \item Ausführen von ausschließlich Unittests
              \item Ausführen von Akzeptanztests
              \item Ausführen der Gesamtheit der Tests
              \item Anpassung der Ziel-IP einer weiteren, für die Testausführung
                    notwendigen Anwendung
          \end{itemize}
    \item uvm.
\end{itemize}

\paragraph{}
Der Kontext der Arbeit an der Software wechselt regelmäßig zwischen Entwicklung und
dem Suchen bzw. Nachvollziehen von potentiellen Softwarefehlern, welche im Produktions-
oder Testbetrieb auftreten.

Um das beobachtete Verhalten der Anwendung korrekt zu interpretieren sind u.a. zwei
Arbeitsschritte häufig zu erledigen:

\begin{itemize}
    \item Prüfen der aktuellen Konfiguration
    \item Anpassen der aktuellen Konfiguration
\end{itemize}