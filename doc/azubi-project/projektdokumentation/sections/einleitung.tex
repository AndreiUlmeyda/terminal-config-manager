\section{Einleitung}
\paragraph{}
Dieses Dokument dient der Dokumentation der betrieblichen Projektarbeit, welche
im Rahmen der Ausbildung zum Fachinformatiker für Anwendungsentwickler durchgeführt
wurde. Die Ausbildung erfolgt unter der Aufsicht der IHK Dresden und im Betrieb
der Check24 Reise Tech Hub und Services GmbH. Der Betrieb ist ebenfalls
Auftraggeber der Projektarbeit.

\paragraph{}
Check24 GmbH ist Betreiber von check24.de, einer Website, auf der verschiedene
Produkte zum Vergleich angeboten werden. Der Auftraggeber betreibt auf dieser
Plattform eine Vergleichsmöglichkeit von Pauschalreisen.

\paragraph{Projektbeschreibung}
Das Projekt dient der Entwicklung eines Software-Tools, das Anwendungsentwickler
bei repetitive Arbeitsabläufen im Umgang mit Konfigurationsdateien unterstützt.
Der Bedarf dafür entstand zwar durch die speziellen Gegebenheiten beim Auftraggeber,
allerdings sind breitere Anwendungsmöglichkeiten denkbar.

\paragraph{Projektziel}
Ziel ist es ein Kommandozeilenprogramm zu entwickeln, welches das Einsehen von
Konfigurationsdateien und das Ändern von Werten innerhalb dieser Dateien vereinfacht.
Die Oberfläche und Nutzung des Programms soll dabei so einfach und übersichtlich
wie möglich gestaltet werden.

\paragraph{Projektbegründung}
Die Zeit die Softwareentwickler in die Bewältigung ihre Aufgaben investieren können
ist aus betrieblicher Sicht eine wichtige, knappe und teure Ressource. Die
Optimierung ihrer Arbeitsprozesse, in diesem Fall die Vereinfachung von repetitiven,
Aufmerksamkeit erfordernden Arbeitsschritten, führt dazu, dass Entwickler ihre
Zeit und Konzentrationsfähigkeit sinnvoller investieren können. Diese Einsparung
kommt wiederum dem Betrieb zugute.

\subsection{Projektabgrenzung}
Das Programm ist vorwiegend als Hilfe für Entwickler gedacht, die eine Vielzahl von
Konfigurationsdateien auf ihrem lokalen Rechner verwalten müssen. Es soll lediglich
per Kommandozeile steuerbar sein und keine zusätzlichen APIs anbieten. Die Einbindung
des Programms in automatische Prozesse, beispielsweise während Deployments, ist
nicht vorgesehen.