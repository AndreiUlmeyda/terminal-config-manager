\subsection{Projektorganisation}
Die Hauptelemente der Verzeichnisstruktur des Projekts sind wie folgt organisiert.

\begin{minted}[bgcolor=codebg]{text}
    \src
    \test
    \doc
    \distribution
    \bin
    \tool
    package.yaml
    README.md
\end{minted}

Die Verzeichnisse enhalten dabei in der angezeigten Reihenfolge

\begin{itemize}
    \item Quellcode des des Programms
    \item Quellcode der Unit- und Integrationstests
    \item generierte Moduldokumentation und Projektdokumentation
    \item Skripte um Softwarepakete für verschiedene Betriebssysteme zu erstellen
    \item kompilierte, ausführbare Dateien
    \item Hilfsskripte, z.B. zur Generierung des Modulabhängigkeitsgraph
\end{itemize}

\paragraph{}
Die Datei \mintinline{bash}{package.yaml} enthält die für Stack notwendigen Metadaten,
beispielsweise Deklarationen der verwendeten Softwarebibliotheken und Konfiguration
des Buildprozesses.

\paragraph{}
Die Datei \mintinline{bash}{README.md} enthält eine kurze Beschreibung des
Programms und eine Nutzungsanleitung zusammen mit offenen Aufgaben und
Zusatzinformationen.