\section{Technologie}
\subsection{Kriterien}
\paragraph{}
Für dieses Projekt bieten sich grundsätzlich alle gängigen Programmiersprachen
an. Während der Umsetzung soll ausgewählten Aspekten der Softwareentwicklung
gesonderte Aufmerksamkeit zukommen.

\paragraph{Korrektheit und Laufzeitstabilität} Es soll auf technischem Weg zum
Einen sichergestellt werden, dass sich das Programm zu jedem Zeitpunkt erwartungsgemäß
und korrekt verhält und zum Anderen, dass Fehlerzustände zur Laufzeit so weit
wie möglich ausgeschlossen werden.

\paragraph{}
Als hauptsächliche Wege dies zu erreichen werden folgende Ansätze gewählt:

\begin{itemize}
    \item Wahl einer Programmiersprache mit strenger Typisierung
    \item Wahl einer kompilierten Programmiersprache mit vergleichsweise starken
          Garantien zum Laufzeitverhalten
    \item Einbeziehung von \gls{Property-Based-Testing} \cite{property-based-testing} in das Konzept der Softwaretests
\end{itemize}

\paragraph{Ausführliche, vom Quellcode abgeleitete Moduldokumentation}
Neben der Projektdokumentation soll eine Dokumentation der einzelnen
Softwaremodule entstehen. Um dem Problem zu begegnen, dass Dokumentation und
Quellcode im Laufe der Entwicklung auseinanderlaufen soll die Moduldokumentation
direkt aus den Quellcodekommentaren generierbar sein.

\subsection{Auswahl}
Als Programmiersprache und Build-System wurden auf Basis der obengenannten Ziele
Haskell \cite{haskell} und Stack \cite{stack} gewählt.

\paragraph{}
Stack bietet neben seiner Hauptaufgabe die Software-Abhängigkeiten des Projekts
zu verwalten und den Buildvorgang zu steuern die Möglichkeit Moduldokumentation
im HTML-Format anhand der Quellcodestrukur und der Quellcodekommentare generieren
während Haskell eine typsichere, kompilierte Programmiersprache mit Unterstützung
für \gls{Property-Based-Testing} darstellt.