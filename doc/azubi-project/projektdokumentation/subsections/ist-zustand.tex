\subsection{IST-Zustand}
Der Großteil dieser Anwendungen besitzt weitläufige Konfigurationsmöglichkeiten
welche ihren Betrieb in verschiedensten Szenarien steuern. Beispiele für
Konfigurationsmöglichkeiten und deren Ausprägungen sind:

\begin{itemize}
    \item Das Loggingverhalten der Anwendung
          \begin{itemize}
              \item Logging gegen die Logverarbeitungssoftware der Produktionsumgebung
              \item Logging gegen eine lokale Instanz der Logverarbeitungssoftware
              \item Logging auf das Dateisystem
              \item Logging mit verschiedenen Logleveln
          \end{itemize}
    \item Die Zieldatenbank der Anwendung \begin{itemize}
              \item Datenbank der Produktionsumgebung
              \item Datenbank der Testumgebung
              \item lokale Datenbank
          \end{itemize}
    \item Die Ausführung von Softwaretests \begin{itemize}
              \item Ausführen von ausschließlich Unittests
              \item Ausführen von Akzeptanztests
              \item Ausführen der Gesamtheit der Tests
              \item Anpassung der Ziel-IP einer weiteren, für die Testausführung
                    notwendigen Anwendung
          \end{itemize}
\end{itemize}

\paragraph{}
Der Kontext der Arbeit an der Software wechselt regelmäßig zwischen Entwicklung und
der Behandlung von Fehlern, welche im Produktions- oder Testbetrieb auftreten.
Um dabei das beobachtete Verhalten der Anwendung korrekt zu interpretieren sind
u.a. zwei Arbeitsschritte häufig zu erledigen:

\begin{itemize}
    \item \textbf{Prüfen} der aktuellen Konfiguration
    \item \textbf{Anpassen} der aktuellen Konfiguration
\end{itemize}

Die Anzahl der an jedem Einzelfall beteiligten Anwendungen und die Anzahl der
Konfigurationsdateien pro Anwendungen führen dazu, dass jeweils viele verschiedene
und weit über das Dateisystem verstreute Dateien relevant sind. Pro Datei und
Einzelfall sind folgende Arbeitsschritte zu erledigen:

\begin{itemize}
    \item Starten der zur Anwendung gehörigen \gls{IDE}
    \item Ermitteln der Konfigurationsdatei
    \item Navigation im Verzeichnisbaum
    \item Öffnen der Datei
    \item Finden des relevanten Eintrags in einer mitunter sehr langen Textdatei
    \item Ermitteln der möglichen Zielwerte
    \item Ändern des Eintrags
    \item Speichern der Datei
\end{itemize}

Diese Einzelschritte, mulipliziert mit der Anzahl an Dateien, stellen eine
große Menge an repetitiven Handlungen dar. Wenn diese erleichtert würden, dann
ließen sich sowohl Zeit und Konzentrationsvermögen einsparen als auch Fehlerpotential
verringern.