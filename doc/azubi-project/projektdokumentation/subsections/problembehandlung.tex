\subsection{Problembehandlung} \label{Problembehandlung}
\paragraph{Fehlende Konfigurationsdatei} Wenn beim Programmstart der Fehler
\begin{minted}[bgcolor=codebg]{bash}
    Error: No config file found at any of the search paths: ...
\end{minted}
auftritt, dann bedeutet das, dass bei der Suche nach Konfigurationsdateien an
keinem der angegebenen Pfade eine Datei gefunden wurde.
\paragraph{Lösung}
Es wird wie in Kapitel \ref{Konfiguration} (Konfiguration) beschrieben eine
Konfigurationsdatei an einem der validen Dateipfade angelegt. Im Dateisystempfad
unter

\begin{minted}[bgcolor=codebg]{bash}
    /usr/share/terminal-config-manager/config.yaml
\end{minted}

befindet sich eine Beispielkonfigurationsdatei, welche als Vorlage genutzt
werden kann:

\begin{minted}[bgcolor=codebg]{bash}
    mkdir ~/.config/terminal-config-manager
    cp  /usr/share/terminal-config-manager/config.yaml \
        ~/.config/terminal-config-manager
\end{minted}

\paragraph{Falsches Konfigurationsdateiformat} \label{missing-config} Wenn beim
Programmstart ein Fehler ähnlich

\begin{minted}[bgcolor=codebg]{text}
    An error occured while parsing the configuration file.
    The details are: ...
\end{minted}

auftritt, dann bedeutet das, dass die erste vom Programm gefundene Konfigurationsdatei
entweder nicht dem YAML-Format \cite{yaml} entspricht und/oder fehlende Elemente aufweist.

\paragraph{Lösung}
Die Fehlermeldung wird weitere Detailinformationen enthalten wie beispielsweise:

\begin{minted}[bgcolor=codebg]{text}
    The top level of the config file
    should be an object named 'config_lines_to_manage'
\end{minted}

anhand derer sich das Problem identifizieren lässt. Im Zweifelsfall sollte dem
Kapitel \ref{Konfiguration} (Konfiguration) bzw. der Problembehandlung für
fehlende Konfigurationsdateien in Kapitel \ref{missing-config} folgend eine valide Konfigurationsdatei
per Hand angelegt werden.

\paragraph{Fehlende Dateizugriffsrechte} Wenn bei der Auswahl eines neuen Werts
der Fehler

\begin{minted}[bgcolor=codebg]{text}
    ...
\end{minted}

auftritt, dann bedeutet das, dass ... .

\paragraph{Lösung}
Herp:

\begin{minted}[bgcolor=codebg]{text}
    ...
\end{minted}

derp.