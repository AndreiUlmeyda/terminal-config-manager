\subsection{Pflichtenheft} \label{Pflichtenheft}
Dieses Kapitel dient als Pflichtenheft des Projekts. Es enthält, aufbauend auf
der Analyse aus dem vorhergehenden Kapitel eine explizite Auflistung der Anforderungen
an die fertige Software inklusive der Abnahmekriterien. Das Projektumfeld und
sonstige Dopplungen mit den vorangestellten Kapiteln werden dabei ausgespart.

\paragraph{Schnittstellenübersicht}
Das zu erstellende Programm ...
\begin{itemize}
    \item erlaubt die Interaktion mittels Tastatureingaben.
    \item erzeugt Ausgaben in einem Terminal(-emulator).
    \item bietet keine sonstigen Schnittstellen an.
\end{itemize}

\paragraph{Lebenszyklusanalyse}
Das Vorgehen bei der Erstellung des Programms folgt dem Wasserfall-Modell. Die
Auslieferung an den Nutzer erfolgt durch die Bereitstellung eines Zugriffs auf
das Hauptverzeichnis des Projekts. Dies kann beispielsweise über das Hosting des
Git-Projekts per Bitbucket oder Github geschehen. Das Projekt gilt nach
Auslieferung als fertiggestellt, jedoch kann über ebengenannte Hosting-Plattformen
Feedback von Nutzern gesammelt werden. Sollten Softwarefehler durch Nutzer
entdeckt werden, dann werden diese so bald wie möglich beseitigt. Ebenfalls
denkbar ist eine Wandlung des Projekts in ein Open-Source-Projekt. Das hätte zur
Folge, dass Fehler der Software in Kollaboration mit Nutzern und auch durch die
Nutzer selbst behoben werden können.

\paragraph{Funktionale Anforderungen}
Das Programm ...
\begin{itemize}
    \item lässt sich von der Kommandozeile aus starten.
    \item erlaubt die Konfiguration der Menge an Zieldateien.
    \item erlaubt die Konfiguration der zu ändernden Werte pro Zieldatei.
    \item kann über eine Datei im YAML-Format konfiguriert werden.
    \item erlaubt mehrere gängige Pfade für Konfigurationsdateien unter Linux.
    \item zeigt vom Start bis zum Beenden des Programms pro Zieldatei eine Textzeile an.
    \item zeigt pro Textzeile die zur Zieldatei gehörige Beschreibung und den aktuellen Wert an.
    \item hebt die aktuell markierte Zeile sichtbar hervor.
    \item erlaubt die Auswahl einer Zeile mit den Tasten "Pfeil hoch" bzw. "Pfeil runter"
    \item erlaubt die Auswahl eines neuen Werts innerhalb Zieldatei der markierten
          Zeile mit den Tasten "Pfeil links" und "Pfeil rechts".
    \item schreibt bei Auswahl eines neuen Werts diesen in die Zieldatei und ersetzt
          damit ausschließlich den vorherigen Wert.
    \item zeigt zu jeder Zeit am unteren Rand des Terminals Hinweise zur Nutzung
          des Programms an.
    \item informiert den Nutzer im Falle einer Fehlkonfiguration.
    \item informiert den Nutzer im Falle eines fehlerhaften Aufrufs des Programms.
    \item lässt sich mit der Taste "q" beenden und kehrt anschließend zur Kommandozeile zurück.
\end{itemize}

\paragraph{Nichtfunktionale Anforderungen}
Das Programm ...
\begin{itemize}
    \item lässt sich auf Debian- und Arch-Linux-Betriebssystemen über die gängigen
          Paketmanager installieren.
    \item ist in seiner Funktionalität mit Unit- und Akzeptanztest abgedeckt.
    \item ist mit ausführlichen Quellcodekommentaren versehen.
    \item ist mit einer navigierbaren Moduldokumentation im HTML-Format versehen.
\end{itemize}

\paragraph{Abnahmekriterien}
Im Projektverzeichnis wird
\begin{minted}[bgcolor=codebg]{bash}
    stack test
\end{minted}
und damit alle automatisierten Tests ausgeführt. Alle Tests müssen erfolgreich
sein. Des Weiteren wird jeder Punkt unter "Funktionale Anforderungen" und
"Nichtfunktionale Anforderungen" händisch geprüft.

\paragraph{Lieferumfang}
Das Programm wird als vollständiges Git-Repository, zusammen mit dem Quellcode,
Moduldokumentation, der ausführbaren Datei und installierbaren Linux-Paketen
ausgeliefert.