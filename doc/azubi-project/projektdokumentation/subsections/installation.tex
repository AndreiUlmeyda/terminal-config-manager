\subsection{Installation}
Es wurden vorkonfigurierte Pakete für sowohl ArchLinux-basierte als auch
Debian-basierte Betriebssysteme bereitgestellt. Alternativ
kann das Programm auch manuell installiert werden.

\begin{center}
    \textbf{Arch-Linux, via PKGBUILD Datei und pacman}
\end{center}

Im Projektverzeichnis unter

\begin{minted}[bgcolor=codebg]{bash}
	/distribution/arch/PKGBUILD
\end{minted}

befindet sich eine Spezifikationsdatei anhand derer das Softwarepaket
erstellt und anschließend installiert werden kann:

\begin{minted}[bgcolor=codebg]{bash}
	cd distribution/arch
	makepkg
	pacman -U terminal-config-manager-1.0.0-1-x86_64.pkg.tar.zst
\end{minted}

Die Deinstallation erfolgt mittels

\begin{minted}[bgcolor=codebg]{bash}
	pacman -R terminal-config-manager
\end{minted}

\begin{center}
    \textbf{Debian, via .deb Datei und dpkg bzw. apt}
\end{center}

Im Projektverzeichnis unter

\begin{minted}[bgcolor=codebg]{bash}
	/distribution/debian/terminal-config-manager.deb
\end{minted}

befindet sich ein Softwarepaket, das mittels \mintinline{bash}{dpkg} oder
\mintinline{bash}{apt} direkt installiert werden kann.

\begin{minted}[bgcolor=codebg]{bash}
	cd distribution/debian
	dpkg --install ./terminal-config-manager.deb
	# apt install ./terminal-config-manager.deb
\end{minted}

Die Deinstallation erfolgt mittels

\begin{minted}[bgcolor=codebg]{bash}
	dpkg --remove terminal-config-manager
	# apt remove terminal-config-manager
\end{minted}

\begin{center}
    \textbf{Alternative, ohne Paketmanager}
\end{center}

Wenn das Programm nicht vom systemeigenen Paketmanager verwaltet werden
soll, dann kann es manuell kompiliert und in einem passenden
Verzeichnis abgelegt werden.

Voraussetzung hierfür ist, dass das Programm \mintinline{bash}{stack} auf dem
System installiert ist.

Im Projektverzeichnis wird mit

\begin{minted}[bgcolor=codebg]{bash}
	stack build --test --copy-bins
\end{minted}

das Programm kompiliert, die Testsuite ausgeführt und die ausführbare Datei im
Projektverzeichnis unter

\begin{minted}[bgcolor=codebg]{bash}
	bin/terminal-config-manager
\end{minted}

abgelegt. Anschließend kann das Programm in ein Verzeichnis kopiert werden, das in die
Systempfadliste eingetragen ist, beispielsweise

\begin{minted}[bgcolor=codebg]{bash}
	cp bin/terminal-config-manager ~/.local/bin
\end{minted}

Die Deinstallation erfolgt mittels

\begin{minted}[bgcolor=codebg]{bash}
	rm ~/.local/bin/terminal-config-manager
	rm <Konfigurationsdateipfad>
\end{minted}