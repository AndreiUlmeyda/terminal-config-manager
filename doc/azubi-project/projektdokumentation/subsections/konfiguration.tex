\subsection{Konfiguration}
Die Zieldateien und -textpassagen müssen vor Ausführung des Programms
über eine Datei im YAML-Format konfiguriert werden.
Das Programm erwartet, dass sich eine solche Datei in einem der folgenden
Verzeichnisse befindet. Die Reihenfolge entspricht der absteigenden Priorität
beim Vorhandensein mehrerer Konfigurationsdateien:

\begin{enumerate}
    \item ./config.yaml
    \item \{\$HOME\}/.config/terminal-config-manager/config.yaml \textbf{(empfohlen)}
    \item \{\$HOME\}/.terminal-config-manager.yaml
\end{enumerate}

Der Dateipfad 1 bezeichnet den Ort der ausführbaren Datei selbst und sollte nur
zu Debugging- oder Entwicklungszecken genutzt werden. Die Pfade 2 und 3 sind
gängige Ablageorte für nutzerspezifische Konfigurationsdateien unter Linux.

\begin{minted}[bgcolor=codebg]{yaml}
config_lines_to_manage:
  - title: Beispieltitel 1
    path: /home/alice/zieldatei.conf
    pattern: "'statspush_enabled' => {{value}},"
    targetValue: "true"
    possibleValues:
      - "true"
      - "false"

  - title: Beispieltitel 2
    path: /home/alice/verzeichnis/weitere-zieldatei.txt
    pattern: "SOFTWARE_ENV={{value}}"
    targetValue: production
    possibleValues:
      - testing
      - staging
      - production
      - local

  - ...
\end{minted}