\subsection{Konfiguration} \label{Konfiguration}
Die Zieldateien und -textpassagen müssen vor Ausführung des Programms
über eine Datei im YAML-Format \cite{yaml} konfiguriert werden.

\paragraph{Verzeichnis}
Das Programm erwartet, dass sich eine solche Datei in einem der folgenden
Verzeichnisse befindet. Die Reihenfolge entspricht der absteigenden Priorität
beim Vorhandensein mehrerer Konfigurationsdateien:

\begin{enumerate}
  \item ./config.yaml
  \item \$\{HOME\}/.config/terminal-config-manager/config.yaml \textbf{(empfohlen)}
  \item \$\{HOME\}/.terminal-config-manager.yaml
\end{enumerate}

Der Dateipfad 1 bezeichnet den Ort der ausführbaren Datei selbst und sollte nur
zu Debugging- oder Entwicklungszecken genutzt werden. Die Pfade 2 und 3 sind
gängige Ablageorte für nutzerspezifische Konfigurationsdateien unter Linux und
sind für die normale Nutzung geeignet.

\newenvironment{code}{\captionsetup{type=listing}}{}
\SetupFloatingEnvironment{listing}{name=Abbildung}

\begin{figure}
  \caption{Beispielaufbau der Konfigurationsdatei}
  \label{fig:sample-config}
  \begin{minted}[bgcolor=codebg]{yaml}
    config_lines_to_manage:
      - title: Beispieltitel 1
        path: /home/alice/zieldatei.conf
        pattern: "'statspush_enabled' => {{value}},"
        targetValue: "true"
        possibleValues:
          - "true"
          - "false"

      - title: Beispieltitel 2
        path: /home/alice/verzeichnis/weitere-zieldatei.txt
        pattern: "SOFTWARE_ENV={{value}}"
        targetValue: production
        possibleValues:
          - testing
          - staging
          - production
          - local

      - ...
  \end{minted}
\end{figure}

\paragraph{Aufbau}
In Abb. \ref{fig:sample-config} ist der Aufbau der Konfigurationsdatei
illustriert.

\begin{itemize}
  \item \begin{minted}[bgcolor=codebg]{yaml}
    config_lines_to_manage:
  \end{minted}
        Dies ist das äußere Element der Konfigurationsdatei und \textbf{muss}
        vorhanden sein. Innerhalb dessen befindet sich eine Liste von Einträgen.
        Jeder Eintrag gehört zu genau einer \gls{Textpassage}, die mithilfe des
        Programms gezielt verändert werden soll.
\end{itemize}



Jeder Eintrag hat folgende Unterelemente:

\begin{itemize}
  \item \begin{minted}[bgcolor=codebg]{yaml}
    title: Beispieltitel 1
  \end{minted}
        Dies ist ein Titel, der frei gewählt werden kann und vom Programm angezeigt
        wird. Idealerweise wird dafür eine sehr kurze Beschreibung des von der
        Ziel-\gls{Textpassage} gesteuerten Anwendungsverhaltens benutzt.
  \item \begin{minted}[bgcolor=codebg]{yaml}
    path: /home/alice/zieldatei.conf
  \end{minted}
        Hier wird der absolute Dateisystempfad der Zieldatei angegeben innerhalb
        derer \gls{Textpassage}n  verändert werden sollen.
  \item \begin{minted}[bgcolor=codebg]{yaml}
    pattern: "'statspush_enabled' => {{value}},"
  \end{minted}
        Dieses Feld enthält ein \gls{Zielmuster}. Mithilfe dieses Musters das einen
        \gls{Platzhalter} enthält wird vom Programm der genaue Ort der Ziel-\gls{Textpassage}
        identifiziert. Das \gls{Zielmuster} sollte sowohl den \gls{Platzhalter}
        als auch an die Ziel-\gls{Textpassage} angrenzenden Text enthalten.
        \textbf{Wenn die Kombination aus \gls{Platzhalter} und angrenzendem Text
          die Ziel-\gls{Textpassage} nicht eindeutig eingrenzt, dann wird vom
          Programm ausschließlich die erste übereinstimmende \gls{Textpassage}
          modifiziert}.
  \item \begin{minted}[bgcolor=codebg]{yaml}
    targetValue: "true"
  \end{minted}
        Der aktuelle Wert der Ziel-\gls{Textpassage}.
        \textbf{Da das Programm in neueren Versionen beim Start den aktuellen
          Wert selbstständig ausliest, wird dieser Konfigurationswert in einer
          der folgenden Versionen entfernt werden. Aktuell muss er jedoch weiterhin
          in der Konfigurationsdatei angegeben werden}.
  \item \begin{minted}[bgcolor=codebg]{yaml}
    possibleValues:
          - "true"
          - "false"
  \end{minted}
        Dies ist die Liste der Werte die mithilfe des Programms ausgewählt und
        anstelle der Ziel-\gls{Textpassage} in die Zieldatei geschrieben werden
        können. \textbf{Wenn sich der aktuelle Wert nicht in dieser Liste befindet,
          dann kann mit dem Programm auch nicht auf den ursprünglichen Wert
          zurückgewechselt werden}.
\end{itemize}