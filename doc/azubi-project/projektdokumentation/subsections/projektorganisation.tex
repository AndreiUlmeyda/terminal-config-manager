\subsection{Projektorganisation}
Die Hauptelemente der Verzeichnisstruktur des Projekts sind wie in Abb. \ref{directory-structure}
im Anhang dargestellt organisiert. Die Verzeichnisse enhalten dabei in der angezeigten Reihenfolge

\begin{itemize}
    \item src: Quellcode des des Programms
    \item test: Quellcode der Unit- und Integrationstests
    \item doc: generierte Moduldokumentation und Projektdokumentation
    \item distribution: Skripte um Softwarepakete für verschiedene Betriebssysteme zu erstellen
    \item bin: kompilierte, ausführbare Dateien
    \item tool: Hilfsskripte, z.B. zur Generierung des Modulabhängigkeitsgraph
\end{itemize}

\paragraph{}
Die Datei \mintinline{bash}{package.yaml} enthält die für Stack notwendigen Metadaten,
beispielsweise Deklarationen der verwendeten Softwarebibliotheken und Konfiguration
des Buildprozesses.

\paragraph{}
Die Datei \mintinline{bash}{README.md} enthält eine kurze Beschreibung des
Programms und eine Nutzungsanleitung zusammen mit offenen Aufgaben und
Zusatzinformationen.