%! Date = 18/07/2022

\documentclass[a4paper,11pt]{scrartcl} % scrartcl scrreprt scrbook

\usepackage{ebgaramond} % gillius ebgaramond
\usepackage{amsmath}
\usepackage[ngerman]{babel}
\usepackage{relsize}
\usepackage{caption}
\usepackage{tabularx}
\usepackage{minted}
\usepackage{pdfpages}
\usepackage[utf8]{inputenc}
\usepackage[automake]{glossaries-extra}
\usepackage[onehalfspacing]{setspace}

\newcommand{\heading}[1]{\multicolumn{1}{c}{#1}}
\setlength{\parindent}{0pt} % disable weird indentation after \begin...\end sections for now

\makeglossaries

\newglossaryentry{latex}
{
	name=latex,
	description={Is a markup language specially suited
			for scientific documents}
}

\newglossaryentry{maths}
{
    name=mathematics,
    description={Mathematics is what mathematicians do}
}

\glsaddall

\author{Adrian Schurz}
\title{Terminal-Config-Manager\\
	Informatik für Anwendungsentwicklung\\
	CHECK24 Tech Hub und Services GmbH\\
	}

\begin{document}

\definecolor{codebg}{rgb}{0.95,0.95,0.95}

\maketitle
\pagenumbering{gobble}
\newpage

\section{Projektantrag}
\includepdf[pages={-}]{../projektantrag/projektantrag_final.pdf}

\section{Nachweisblatt}
\paragraph{}

\newpage
\tableofcontents
\newpage

% Textteil
\pagenumbering{arabic}

\section{Problemstellung}
Lorem ipsum \gls{latex} derpledoo \gls{maths}

\section{Literaturverzeichnis}

\section{Anlagen}

\section{Kundendokumentation}

\subsection{Beschreibung}
Terminal-Config-Manager ist ein Linux-Programm mit welchem
Passagen innerhalb meherer Textdateien schnell zwischen einer Reihe
vorkonfigurierter Passagen umgeschalten werden können.

Der Hauptanwendungsfall ist die effiziente  Manipulation von
Konfigurationsdateien von Softwareanwendungen, die häufig angepasst
werden müssen.

\subsection{Installation} \label{Installation}
Es wurden vorkonfigurierte Pakete für sowohl ArchLinux-basierte als auch
Debian-basierte Betriebssysteme bereitgestellt. Alternativ
kann das Programm auch manuell installiert werden.

\begin{center}
	\textbf{Arch-Linux, via PKGBUILD Datei und pacman}
\end{center}

Im Projektverzeichnis unter

\begin{minted}[bgcolor=codebg]{bash}
	/distribution/arch/PKGBUILD
\end{minted}

befindet sich eine Spezifikationsdatei anhand derer das Softwarepaket
erstellt und anschließend installiert werden kann:

\begin{minted}[bgcolor=codebg]{bash}
	cd distribution/arch
	makepkg
	pacman -U terminal-config-manager-1.0.0-1-x86_64.pkg.tar.zst
\end{minted}

Die Deinstallation erfolgt mittels

\begin{minted}[bgcolor=codebg]{bash}
	pacman -R terminal-config-manager
\end{minted}

\begin{center}
	\textbf{Debian, via .deb Datei und dpkg bzw. apt}
\end{center}

Im Projektverzeichnis unter

\begin{minted}[bgcolor=codebg]{bash}
	/distribution/debian/terminal-config-manager.deb
\end{minted}

befindet sich ein Softwarepaket, das mittels \mintinline{bash}{dpkg} oder
\mintinline{bash}{apt} direkt installiert werden kann.

\begin{minted}[bgcolor=codebg]{bash}
	cd distribution/debian
	dpkg --install ./terminal-config-manager.deb
	# apt install ./terminal-config-manager.deb
\end{minted}

Die Deinstallation erfolgt mittels

\begin{minted}[bgcolor=codebg]{bash}
	dpkg --remove terminal-config-manager
	# apt remove terminal-config-manager
\end{minted}

\begin{center}
	\textbf{Alternative, ohne Paketmanager}
\end{center}

Wenn das Programm nicht vom systemeigenen Paketmanager verwaltet werden
soll, dann kann es manuell kompiliert und in einem passenden
Verzeichnis abgelegt werden.

Voraussetzung hierfür ist, dass das Programm \mintinline{bash}{stack} auf dem
System installiert ist.

Im Projektverzeichnis wird mit

\begin{minted}[bgcolor=codebg]{bash}
	stack build --test --copy-bins
\end{minted}

das Programm kompiliert, die Testsuite ausgeführt und die ausführbare Datei im
Projektverzeichnis unter

\begin{minted}[bgcolor=codebg]{bash}
	bin/terminal-config-manager
\end{minted}

abgelegt. Anschließend kann das Programm in ein Verzeichnis kopiert werden, das in die
Systempfadliste eingetragen ist, beispielsweise

\begin{minted}[bgcolor=codebg]{bash}
	cp bin/terminal-config-manager ~/.local/bin
\end{minted}

Die Deinstallation erfolgt mittels

\begin{minted}[bgcolor=codebg]{bash}
	rm ~/.local/bin/terminal-config-manager
	rm <Konfigurationsdateipfad>
\end{minted}

\subsection{Konfiguration}
Die Zieldateien und -textpassagen müssen vor Ausführung des Programms
über eine Datei im YAML-Format konfiguriert werden.
Das Programm erwartet, dass sich eine solche Datei in einem der folgenden
Verzeichnisse befindet. Die Reihenfolge entspricht der absteigenden Priorität
beim Vorhandensein mehrerer Konfigurationsdateien:

\begin{enumerate}
	\item ./config.yaml
	\item \{\$HOME\}/.config/terminal-config-manager/config.yaml \textbf{(empfohlen)}
	\item \{\$HOME\}/.terminal-config-manager.yaml
\end{enumerate}

Der Dateipfad 1 bezeichnet den Ort der ausführbaren Datei selbst und sollte nur
zu Debugging- oder Entwicklungszecken genutzt werden. Die Pfade 2 und 3 sind
gängige Ablageorte für nutzerspezifische Konfigurationsdateien unter Linux.

\begin{minted}[bgcolor=codebg]{yaml}
config_lines_to_manage:
  - title: Beispieltitel 1
    path: /home/alice/zieldatei.conf
    pattern: "'statspush_enabled' => {{value}},"
    targetValue: "true"
    possibleValues:
      - "true"
      - "false"

  - title: Beispieltitel 2
    path: /home/alice/verzeichnis/weitere-zieldatei.txt
    pattern: "SOFTWARE_ENV={{value}}"
    targetValue: production
    possibleValues:
      - testing
      - staging
      - production
      - local

  - ...
\end{minted}

In Abb

\subsection{Benutzung} \label{Benutzung}
Das Programm wird nach erfolgreicher Installation mit dem Befehl

\begin{minted}[bgcolor=codebg]{bash}
    terminal-config-manager
\end{minted}

von der Kommandozeile aus gestartet.

\subsection{Problembehandlung} \label{Problembehandlung}
\paragraph{Fehlende Konfigurationsdatei} Wenn beim Programmstart der Fehler
\begin{minted}[bgcolor=codebg]{bash}
    Error: No config file found at any of the search paths: ...
\end{minted}
auftritt, dann bedeutet das, dass bei der Suche nach Konfigurationsdateien an
keinem der angegebenen Pfade eine Datei gefunden wurde.
\paragraph{Lösung}
Es wird wie in Kapitel \ref{Konfiguration} (Konfiguration) beschrieben eine
Konfigurationsdatei an einem der validen Dateipfade angelegt. Im Dateisystempfad
unter

\begin{minted}[bgcolor=codebg]{bash}
    /usr/share/terminal-config-manager/config.yaml
\end{minted}

befindet sich eine Beispielkonfigurationsdatei, welche als Vorlage genutzt
werden kann:

\begin{minted}[bgcolor=codebg]{bash}
    mkdir ~/.config/terminal-config-manager
    cp  /usr/share/terminal-config-manager/config.yaml \
        ~/.config/terminal-config-manager
\end{minted}

\paragraph{Falsches Konfigurationsdateiformat} \label{missing-config} Wenn beim
Programmstart ein Fehler ähnlich

\begin{minted}[bgcolor=codebg]{text}
    An error occured while parsing the configuration file.
    The details are: ...
\end{minted}

auftritt, dann bedeutet das, dass die erste vom Programm gefundene Konfigurationsdatei
entweder nicht dem YAML-Format \cite{yaml} entspricht und/oder fehlende Elemente aufweist.

\paragraph{Lösung}
Die Fehlermeldung wird weitere Detailinformationen enthalten wie beispielsweise:

\begin{minted}[bgcolor=codebg]{text}
    The top level of the config file
    should be an object named 'config_lines_to_manage'
\end{minted}

anhand derer sich das Problem identifizieren lässt. Im Zweifelsfall sollte dem
Kapitel \ref{Konfiguration} (Konfiguration) bzw. der Problembehandlung für
fehlende Konfigurationsdateien in Kapitel \ref{missing-config} folgend eine valide Konfigurationsdatei
per Hand angelegt werden.

\paragraph{Fehlende Dateizugriffsrechte} Wenn bei der Auswahl eines neuen Werts
der Fehler

\begin{minted}[bgcolor=codebg]{text}
    ...
\end{minted}

auftritt, dann bedeutet das, dass ... .

\paragraph{Lösung}
Herp:

\begin{minted}[bgcolor=codebg]{text}
    ...
\end{minted}

derp.

\printglossaries

\end{document}