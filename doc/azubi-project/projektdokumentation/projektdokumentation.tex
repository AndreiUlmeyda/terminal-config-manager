%! Date = 18/07/2022

\documentclass[a4paper,11pt]{scrartcl} % scrartcl scrreprt scrbook

\usepackage{ebgaramond} % gillius ebgaramond
\usepackage{amsmath}
\usepackage[ngerman]{babel}
\usepackage{relsize}
\usepackage{caption}
\usepackage{tabularx}
\usepackage{minted}
\usepackage[onehalfspacing]{setspace}

\newcommand{\heading}[1]{\multicolumn{1}{c}{#1}}
\setlength{\parindent}{0pt} % disable weird indentation after \begin...\end sections for now

\author{Adrian Schurz}
\title{Terminal-Config-Manager\\
	Informatik für Anwendungsentwicklung\\
	Check24 Reisebla\\ % TODO
	}

\begin{document}

\definecolor{codebg}{rgb}{0.95,0.95,0.95}

\maketitle
\pagenumbering{gobble}
\newpage

\section{Projektantrag}
\paragraph{}

\section{Nachweisblatt}
\paragraph{}

\newpage
\tableofcontents
\newpage

% Textteil
\pagenumbering{arabic}

\section{Problemstellung}
\paragraph{}

% TODO Literaturverzeichnis

\section{Glossar}

% evtl. Abkürzungsverzeichnis

\section{Anlagen}

\section{Kundendokumentation}

\subsection{Beschreibung}
Terminal-Config-Manager ist ein Linux-Programm mit welchem
Passagen innerhalb meherer Textdateien schnell zwischen einer Reihe
vorkonfigurierter Passagen umgeschalten werden können.

Der Hauptanwendungsfall ist die effiziente  Manipulation von
Konfigurationsdateien von Softwareanwendungen, die häufig angepasst
werden müssen.

\subsection{Installation}
Es wurden vorkonfigurierte Pakete für sowohl ArchLinux-basierte als auch
Debian-basierte Betriebssysteme bereitgestellt. Alternativ
kann das Programm auch manuell installiert werden.

\paragraph{Arch-Linux, via PKGBUILD Datei und pacman}\mbox{}\\
Im Projektverzeichnis unter

\begin{minted}[bgcolor=codebg]{bash}
	/distribution/arch/PKGBUILD
\end{minted}

befindet sich eine Spezifikationsdatei anhand derer das Softwarepaket
erstellt und anschließend installiert werden kann:

\begin{minted}[bgcolor=codebg]{bash}
	cd distribution/arch
	makepkg
	pacman -U terminal-config-manager-1.0.0-1-x86_64.pkg.tar.zst
\end{minted}

Die Deinstallation erfolgt mittels

\begin{minted}[bgcolor=codebg]{bash}
	pacman -R terminal-config-manager
\end{minted}

\paragraph{Debian, via .deb Datei und dpkg bzw. apt}\mbox{}\\
Im Projektverzeichnis unter

\begin{minted}[bgcolor=codebg]{bash}
	/distribution/debian/terminal-config-manager.deb
\end{minted}

befindet sich ein Softwarepaket, das mittels \mintinline{bash}{dpkg} oder
\mintinline{bash}{apt} direkt installiert werden kann.

\begin{minted}[bgcolor=codebg]{bash}
	cd distribution/debian
	dpkg --install ./terminal-config-manager.deb
	# apt install ./terminal-config-manager.deb
\end{minted}

Die Deinstallation erfolgt mittels

\begin{minted}[bgcolor=codebg]{bash}
	dpkg --remove terminal-config-manager
	# apt remove terminal-config-manager
\end{minted}

\paragraph{Alternative, ohne Paketmanager}\mbox{}\\
Wenn das Programm nicht vom systemeigenen Paketmanager verwaltet werden
soll, dann kann es manuell kompiliert und in einem passenden
Verzeichnis abgelegt werden.

Voraussetzung hierfür ist, dass das Programm \mintinline{bash}{stack} auf dem
System installiert ist.

Im Projektverzeichnis wird mit

\begin{minted}[bgcolor=codebg]{bash}
	stack build --test --copy-bins
\end{minted}

das Programm kompiliert, die Testsuite ausgeführt und die ausführbare Datei im
Projektverzeichnis unter

\begin{minted}[bgcolor=codebg]{bash}
	bin/terminal-config-manager
\end{minted}

abgelegt. Anschließend kann das Programm in ein Verzeichnis kopiert werden, das in die
Systempfadliste eingetragen ist, beispielsweise

\begin{minted}[bgcolor=codebg]{bash}
	cp bin/terminal-config-manager ~/.local/bin
\end{minted}

Die Deinstallation erfolgt mittels

\begin{minted}[bgcolor=codebg]{bash}
	rm ~/.local/bin/terminal-config-manager
	rm <Konfigurationsdateipfad>
\end{minted}

\subsection{Konfiguration}
Die Zieldateien und -textpassagen müssen vor Ausführung des Programms
über eine Datei im YAML-Format konfiguriert werden.
Das Programm erwartet, dass sich eine solche Datei in einem der folgenden
Verzeichnisse befindet. Die Reihenfolge entspricht der absteigenden Priorität
beim Vorhandensein mehrerer Konfigurationsdateien:

\begin{enumerate}
	\item ./config.yaml
	\item \{\$HOME\}/.config/terminal-config-manager/config.yaml \textbf{(empfohlen)}
	\item \{\$HOME\}/.terminal-config-manager.yaml
\end{enumerate}

Der Dateipfad 1 bezeichnet den Ort der ausführbaren Datei selbst und sollte nur
zu Debugging- oder Entwicklungszecken genutzt werden. Die Pfade 2 und 3 sind
gängige Ablageorte für nutzerspezifische Konfigurationsdateien unter Linux.

\begin{minted}[bgcolor=codebg]{yaml}
config_lines_to_manage:
  - title: Beispieltitel 1
    path: /home/alice/zieldatei.conf
    pattern: "'statspush_enabled' => {{value}},"
    targetValue: "true"
    possibleValues:
      - "true"
      - "false"

  - title: Beispieltitel 2
    path: /home/alice/verzeichnis/weitere-zieldatei.txt
    pattern: "SOFTWARE_ENV={{value}}"
    targetValue: production
    possibleValues:
      - testing
      - staging
      - production
      - local

  - ...
\end{minted}

\subsection{Benutzung}

\subsection{Problembehandlung}
% TODO ausformulieren
Fehler, keine Config -> die Datei fehlt an einem der üblichen Zielorte.
-> Beispielconfig nehmen
Fehler, Config parsing -> Die erste gefundene Configdatei ist falsch
formatiert und/oder unvollständig -> Format prüfen mittels Schema
oder mit Beispiel-Config vergleichen oder anhand der Parsing-Fehlermeldung
den Fehler beheben.

\end{document}