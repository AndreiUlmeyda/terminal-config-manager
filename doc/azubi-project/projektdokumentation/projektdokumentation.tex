%! Date = 18/07/2022

\documentclass[a4paper,11pt]{scrartcl} % scrartcl scrreprt scrbook

\usepackage{ebgaramond} % gillius ebgaramond
\usepackage{amsmath}
\usepackage[ngerman]{babel}
\usepackage{relsize}
\usepackage{caption}
\usepackage{tabularx}
\usepackage{minted}
\usepackage[onehalfspacing]{setspace}

\newcommand{\heading}[1]{\multicolumn{1}{c}{#1}}

\author{Adrian Schurz}
\title{Terminal-Config-Manager\\
	Informatik für Anwendungsentwicklung\\
	Check24 Reisebla\\ % TODO
	}

\begin{document}

\maketitle
\pagenumbering{gobble}
\newpage

\section{Projektantrag}
\paragraph{}
Lorem ipsum dolor sit amet, consectetur adipiscing elit. Aenean ac velit vel metus malesuada egestas. Fusce lorem purus, condimentum id elit id, lobortis suscipit risus. Quisque vehicula dui sit amet risus laoreet tristique. Morbi ac tincidunt dolor. Sed faucibus arcu vel rhoncus tempor. Phasellus ante tellus, tempus sed molestie et, feugiat cursus felis. Mauris vitae leo id justo tempor rhoncus eget ac metus. Proin faucibus et urna vel lobortis. Morbi et velit ac odio scelerisque posuere. Aenean rutrum tristique augue, id placerat diam vestibulum ac. Etiam pulvinar tortor eu auctor molestie.

\section{Nachweisblatt}
\paragraph{}
Lorem ipsum dolor sit amet, consectetur adipiscing elit. Aenean ac
velit vel metus malesuada egestas. Fusce lorem purus, condimentum
id elit id, lobortis suscipit risus. Quisque vehicula dui sit amet
risus laoreet tristique. Morbi ac tincidunt dolor. Sed faucibus
arcu vel rhoncus tempor. Phasellus ante tellus, tempus sed molestie et, feugiat cursus felis. Mauris vitae leo id justo tempor rhoncus eget ac metus. Proin faucibus et urna vel lobortis. Morbi et velit ac odio scelerisque posuere. Aenean rutrum tristique augue, id placerat diam vestibulum ac. Etiam pulvinar tortor eu auctor molestie.

\newpage
\tableofcontents
\newpage

% Textteil
\pagenumbering{arabic}

\section{Problemstellung}
\paragraph{}
Lorem ipsum dolor sit amet, consectetur adipiscing elit. Aenean ac
velit vel metus malesuada egestas. Fusce lorem purus, condimentum
id elit id, lobortis suscipit risus. Quisque vehicula dui sit amet
risus laoreet tristique. Morbi ac tincidunt dolor. Sed faucibus
arcu vel rhoncus tempor. Phasellus ante tellus, tempus sed molestie
et, feugiat cursus felis. Mauris vitae leo id justo tempor rhoncus
eget ac metus. Proin faucibus et urna vel lobortis. Morbi et velit
ac odio scelerisque posuere. Aenean rutrum tristique augue, id
placerat diam vestibulum ac. Etiam pulvinar tortor eu auctor
molestie.

% TODO Literaturverzeichnis

\section{Glossar}

% evtl. Abkürzungsverzeichnis

\section{Anlagen}

\section{Kundendokumentation}

\subsection{Beschreibung}
Terminal-Config-Manager ist ein Linux-Programm mit welchem
Passagen innerhalb meherer Textdateien schnell zwischen einer Reihe
vorkonfigurierter Passagen umgeschalten werden können.

Der Hauptanwendungsfall ist die effiziente  Manipulation von
Konfigurationsdateien von Softwareanwendungen, die häufig angepasst
werden müssen.

\subsection{Installation}
Es wurden vorkonfigurierte Pakete für sowohl ArchLinux-basierte als auch
Debian-basierte Betriebssysteme bereitgestellt. Alternativ
kann das Programm auch manuell installiert werden.

\paragraph{Arch-Linux, via PKGBUILD Datei und pacman:}
Im Projektverzeichnis unter

\begin{minted}{bash}
	/distribution/arch/PKGBUILD
\end{minted}

befindet sich ein eine Spezifikationsdatei anhand derer das Softwarepaket
erstellt und anschließend installiert werden kann:

\begin{minted}{bash}
	cd distribution/arch
	makepkg
	pacman -U terminal-config-manager-x86_64.pkg.tar.zst
\end{minted}

Die Deinstallation erfolgt mittels

\begin{minted}{bash}
	pacman -R terminal-config-manager
\end{minted}

\paragraph{Debian, via .deb Datei und dpkg bzw. apt:}
Im Projektverzeichnis unter

\begin{minted}{bash}
	/distribution/debian/terminal-config-manager.deb
\end{minted}

befindet sich ein Softwarepaket, das mittels \mintinline{bash}{dpkg} oder
\mintinline{bash}{apt} installiert werden kann.

\begin{minted}{bash}
	cd distribution/debian
	dpkg --install ./terminal-config-manager.deb
	# apt install ./terminal-config-manager.deb
\end{minted}

Die Deinstallation erfolgt mittels

\begin{minted}{bash}
	dpkg --remove terminal-config-manager
	# apt remove terminal-config-manager
\end{minted}

\paragraph{Alternative, ohne Paketmanager:}
Wenn das Programm nicht vom systemeigenen Paketmanager verwaltet werden
soll, dann kann es manuell kompiliert und in einem passenden
Verzeichnis abgelegt werden.

Voraussetzung hierfür ist, dass das Programm \mintinline{bash}{stack} auf dem
System installiert ist.

Im Projektverzeichnis wird mit

\begin{minted}{bash}
	stack build --test --copy-bins
\end{minted}

das Programm kompiliert, die Testsuite ausgeführt und die ausführbare Datei im
Projektverzeichnis unter

\begin{minted}{bash}
	bin/terminal-config-manager
\end{minted}

abgelegt. Anschließend kann das Programm in ein Verzeichnis kopiert werden, das in die
Systempfadliste eingetragen ist, beispielsweise

\begin{minted}{bash}
	cp bin/terminal-config-manager ~/.local/bin
\end{minted}

Die Deinstallation erfolgt mittels

\begin{minted}{bash}
	rm ~/.local/bin/terminal-config-manager
	rm <Konfigurationsdateipfad>
\end{minted}

\subsection{Konfiguration}

\subsection{Benutzung}

\subsection{Problembehandlung}
% TODO ausformulieren
Fehler, keine Config -> die Datei fehlt an einem der üblichen Zielorte.
-> Beispielconfig nehmen
Fehler, Config parsing -> Die erste gefundene Configdatei ist falsch
formatiert und/oder unvollständig -> Format prüfen mittels Schema
oder mit Beispiel-Config vergleichen oder anhand der Parsing-Fehlermeldung
den Fehler beheben.

\end{document}