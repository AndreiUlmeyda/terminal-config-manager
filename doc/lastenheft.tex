%! Date = 18/07/2022

\documentclass[a4paper,11pt]{article}

\usepackage{amsmath}
\usepackage[ngerman]{babel}
\usepackage{relsize}
\usepackage{caption}
\usepackage{tabularx}
\usepackage{minted}

\author{Adrian Schurz}
\title{Lastenheft\\[0.2em]\smaller{}Terminal-Config-Manager (Arbeitstitel)}

\begin{document}

\maketitle
\pagenumbering{gobble}
\newpage
\pagenumbering{arabic}
\tableofcontents

\section{Einführung}
\paragraph{}
Dieses Dokument dient als Lastenheft im Rahmen der externen Zulassung
zur Abschlussprüfung einer Ausbildung zum Fachinformatiker für
Anwendungsentwicklung bei der IHK Dresden. Es beschreibt das Ergebnis der
Anforderungsanalyse des gewählten Softwareprojekts. Dazu gehören u.a. die
Motivation des Projekts, die Anforderungen sowie Nicht-Anforderungen an
die Software als auch die prüfbaren Kriterien anhand derer die Abnahme des
fertigen Produkts zu erfolgen hat.

\section{Motivation}
\paragraph{}
Bei der Arbeit an Software-Diensten, welche eng mit einer Vielzahl anderer
solcher Dienste über mehrere Umgebungen hinweg interagieren, ergaben sich
für mich zwei häufige Anwendungsfälle (\textbf{A} und \textbf{B}) beim Umgang mit
Konfigurationsdateien.
\paragraph{A}
Das Nachschlagen gezielter Werte in den Konfigurationsdateien
einzelner Dienste oder Hilfsprogramme auf meinem lokalen, zur Entwicklung
genutzten Computer.
\paragraph{B}
Das Umschalten einer kleinen Menge häufig zu ändernder Konfigurationswerte
zwischen verschiedenen Werten. Die Anzahl verschiedener Zielwerte ist dabei
meist sehr begrenzt.
\paragraph{}
Diese beiden häufigen Arbeitsvorgänge möchte ich so gut wie möglich beschleunigen.

\subsection{IST-Zustand}
\subsection{SOLL-Zustand}

\begin{center}
	\begin{table}[h]
		\caption{Nutzungsszenarien und Vergleich der Arbeitsabläufe}
	\end{table}
\end{center}

\section{Nicht-Ziele}
\paragraph{A}
Das automatische Ausführen von Folgeaktionen nach Änderung eines Wertes. \textbf{Begründung:}
Die möglichen Arten von Folgeaktionen, bspw. das (Neu-)Starten von Anwendungen,
das Löschen von Cachedateien, usw., sind so vielfältig, dass zu diesem Zweck vom
Programm beliebige Shellskripte o.ä. ausgeführt werden müssten. Das erhöht zum Einen
unnötigerweise die Komplexität des Programms und zum Anderen wird die Betrachtung von
Sicherheitsaspekten wesentlich schwieriger. Um Folgeaktionen auszuführen exisiteren
eine Vielzahl spezialisierter Tools die jeweils einem begrenzten Aufgabenbereich
gerecht werden und sich gut miteinander kombinieren lassen.

\begin{figure}[!ht]
	\caption[cdba]{Beispiel zum Ausführen einer beliebigen Folgeaktion nach
		Änderung an einer Konfigurationsdatei in bash\footnotemark mittels entr.\footnotemark}
	\begin{minted}{bash}
ls <Konfigurationsdateipfad> | entr Folgeaktion.sh
\end{minted}
\end{figure}
\footnotetext[1]{https://www.gnu.org/software/bash/}
\footnotetext[2]{https://eradman.com/entrproject/}

\paragraph{B}
Das gleichzeitige Ändern mehrerer Konfigurationswerte in festgelegten Kombinationen
oder "Profilen". \textbf{Begründung:} Wenn mehrere voneinander abhängige Konfigurationswerte
existieren, die beim Wechsel hin zu einem anderen Nutzungsszenario immer gleichzeitig
geändert werden müssen, dann wäre es sinnvoller das Konfigurationsschema der jeweiligen
Anwendung anzupassen, sodass das Nutzungszenario von einem einzelnen Wert bestimmt wird.

\paragraph{C}
Ein grafisches Benutzerinterface. Dazu gehören Webinterfaces.


\end{document}

% TODO Nichtziel B besser begründen
% TODO Überlegung: Wert bei jedem switch ändern vs erst nach Enter